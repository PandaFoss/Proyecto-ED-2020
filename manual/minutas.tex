\documentclass[11pt]{article}

    \title{\textbf{Minutas del proyecto}}
    \date{}

    \addtolength{\topmargin}{-3cm}
    \addtolength{\textheight}{3cm}
    \usepackage{fancyhdr}
    \usepackage{geometry}
    \geometry{
        a4paper,
        total={170mm,257mm},
        left=20mm,
        top=20mm,
    }
	\usepackage{graphicx}
\begin{document}

% Portada
\maketitle
\thispagestyle{empty}
% Minutas en ítems
\begin{enumerate}
	\item Si bien la clave de acceso pretende funcionar mediante el formato \(AXA'A'\), donde \(A\) es el apellido del usuario, \(X\) es la letra \("X"\) y \(A'\) es el apellido pero invertido, el software implementado no corrobora que \(A\) sea efectivamente un apellido. Aún así, la eficacia es la misma.
	\item Surgieron dudas con respecto a la implementacion de algunos metodos del TDA Deque. De todas formas no son métodos relevantes para el funcionamiento del programa del proyecto.
	\item Utilizamos una Cola con Prioridad con Heap ya que los unicos metodos que usamos fueron \texttt{insert()} y \texttt{min()}, y este tipo de implementacion nos permite obtener la mayor eficiencia en ambos, tomando \texttt{insert()} con \(O(log_2(n))\) y \texttt{min()} con \(O(log_2(n))\).
\end{enumerate}
\end{document}
